%************************************************
\chapter{Una Introducción a Scala}
\label{ch:scalaintro}
%************************************************

El nombre \textsc{Scala} es una concatenación de dos palabras, \emph{Scalable
  Language}. A continuación se enumeran algunas de las razonas por las que se ha
elegido este lenguaje, la lista completa puede encontrarse en
\citeauthor{Dean2015} \cite{Dean2015}.

\section{¿Por qué \textsc{Scala}?}
\label{sec:whyscala}

Las principales características de \textsc{Scala} que lo hacen un buen candidato
para este trabajo son las siguientes:

\begin{description}
\item[Paradigma mixto --- Programación Orientada a Objetos:] \textsc{Scala}
  soporta al completo el paradigma de la orientación a objetos. Además, mejora
  el modelo de objetos proporcionado por \textsc{Java} con la introducción de
  \textsc{Traits}, un modo muy claro de implementar tipos mediante composiciones
  mixtas. Todo es un objeto en \textsc{Scala}, incluso los tipos numéricos.
\item [Paradigma mixto --- Programación Funcional:] De igual modo, \textsc{Scala}
  soporta al completo \acfi{FP}\acused{FP}\graffito{\ac{FP}: Programación
    Funcional}. En los últimos años, la \ac{FP} ha resurgido como una de las
  mejores herramientas para pensar en problemas de concurrencia, \emph{Big Data}
  y en general para escribir código \enhancement{correctness}correcto. Este
  código correcto, conciso y potente se logra mediante el uso de valores
  inmutables, funciones de primera clase, funciones sin efectos colaterales,
  funciones de ``orden superior'' y colecciones de funciones.
\item [Sintaxsis breve, elegante y flexible:] Expresiones que pueden llegar a
  ser demasiado extensas en \textsc{Java} se hacen concisas en \textsc{Scala}.
\item [Arquitectura Scalable:] \textsc{Scala} permite escribir desde
  \emph{scripts} pequeños, que son interpretados, hasta aplicaciones
  distribuidas de gran envergadura. Hay cuatro mecanismos inherentes al lenguaje
  permitiendo esta escalabilidad: 1) composiciones mixtas mediante
  \textsc{Traits}, 2) miembros de tipo abstracto y genéricos; 3) anidamiento de
  clases y 4) tipos explícitos \textsc{self}.
  
\end{description}

%*****************************************
%*****************************************
%*****************************************
%*****************************************
%*****************************************
