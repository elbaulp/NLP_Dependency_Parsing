%*****************************************
\chapter{Parseo de dependencias en Español}\label{ch:depparsing}
% *****************************************

\todo[inline]{Desribir de forma más detallada el Dep Parsing, mencionar state of
  the art (3/4 papers), entre ellos el implementado}

\todo[inline]{Sección con algoritmo implementado, reiterando sección anterior
  pero con lujo de detalles (Teóricos y código)}

% Los titulas como te digo y extiendes un poco lo que se hace en cada uno de ellos
% (3-4 frases por objetivo):

% - Revisión bibliográfica del estado del arte y antecedentes del dep parsing en
% español. (Enlazas con capitulos de la memoria de antecedentes y dep parsing)

% - Elección y análisis de requerimientos de un procedimiento apropiado de dep
% parsing a partir de la revisión anterior y diseño del mismo para el lenguaje de
% programación Scala (Enlazas con la explicación teórica del algoritmo que has
% implementado y con un capítulo de introducción a Scala).

% - Implementación y procesos de prueba del algoritmo escogido. (Capítulos de
% implementación del algoritmo y de tests de ingeniería del software)

% - Evaluación del algoritmo, comparación y discusión de resultados obtenidos en
% casos prácticos (Capítulo sobre resultados obtenidos, comparación con los
% resultados originales y otros disponibles y último capítulo de conclusiones).

% Los objetivos mencionan la resolución del trabajo como uno de ellos, pero sin
% entrar en detalle tal y como en el que se entra en los capítulos
% correspondientes. La sección de objetivos se limita a enumerarlos y explicar
% brevemente en qué consisten. En la Resolución del trabajo, te explicas con todo
% lujo de detalles técnicos y científicos sobre los métodos desarrollados.

Como se comentó en la \autoref{sec:currentnlp}, el \ac{DP} establece relaciones
entre palabras principales y sus modificadores. En este capítulo se revisarán
los métodos existentes para tal fin, así como el desarrollado en el trabajo.

\section{Revisión bibliográfica del estado del arte y antecedentes del parseo de
  dependencias en Español}
\label{sec:revBibliografica}

Comenzaremos con el trabajo de \citeauthor{ballesteros2016}
\cite{ballesteros2016}

\section{Elección y análisis de requerimientos de un procedimiento apropiado de
  parseo de dependencias y diseño para Scala}
\label{sec:analReq}

\section{Implementación y procesos de prueba del algoritmo escogido}
\label{sec:impl}

\section{Evaluación del algoritmo, comparación y discusión de resultados
  obtenidos en casos prácticos}
\label{sec:eval}




%*****************************************
%*****************************************
%*****************************************
%*****************************************
%*****************************************
