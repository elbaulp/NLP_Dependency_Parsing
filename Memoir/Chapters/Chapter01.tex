%************************************************
\chapter{Motivación e Introducción}\label{ch:introduction}
% ************************************************

\section{¿Qué es el Procesamiento del Lenguaje Natural?}
\label{sec:whatisnlp}

El lenguaje natural se refiere a cualquier lenguaje hablado por un humano, (\eg
Inglés, Castellano o Chino). El \ac{PNL} o \ac{NLP} en inglés es un campo de la
ciencia de la computación e ingeniería desarrollado a partir del estudio del
lenguaje y la computación lingüistica dentro del campo de la \ac{IA}. Los
objetivos del \ac{NLP} son diseñar y construir aplicaciones que faciliten la
interacción humana con la máquinas y otros dispositivos mediante el uso del
lenguaje natural. Dentro del amplio campo del \ac{NLP} podemos distinguir las
siguientes áreas principales:

\begin{description}
  \item[QAS:] \ac{QAS}, o \ac{SRA} en castellano. En estos sistemas se pretende
    remplazar a los usuales buscadores en los que introducimos un texto para
    obtener algún tipo de respuesta a una pregunta. Por ejemplo, si quisieramos
    saber a qué hora abre un centro comercial, bastaría con hablarle al sistema
    en lenguaje natural -- nuestro lenguaje natural, ya sea Inglés, Alemán o
    Castellano y el sistema nos daría respuesta a nuestra pregunta. Aunque ya
    existen este tipo de sistemas (\eg \emph{Siri, Cortana\dots}) están aún en
    una situación muy precaria, ya que ninguno entiende por completo el lenguaje
    natural, solo un subconjunto de frases clave.
  \item[Resúmenes:] Este área incluye aplicaciones que puedan, basándose en una
    colección de documentos, dar como salida un resumen coherente del contenido
    de los mismos. Otra de las posibles aplicaciones sería generar
    presentaciones a partir de dichos documentos.
  \item[Traducción:] Esta fue la principal área de investigación en el campo del
    \ac{NLP}. Como claro ejemplo tenemos el traductor de Google, mejorando día a
    día. Sin embargo, un traductor realmente útil sería aquel que consiga
    traducir en tiempo real una frase que le dictemos mientras decidimos qué
    línea de autobús debemos coger para llegar a tiempo a una conferencia en
    Zurich.
  \item[Reconocimiento de voz:] Una de las tareas más difíciles en \ac{NLP}. Aún
    así, se han conseguido grandes avances en la construcción de modelos que
    pueden usarse en el teléfono móvil o en el ordenador. Estos modelos son
    capaces de reconocer expresiones del lenguaje hablado como preguntas y
    comandos. Desafortunadamente, los sistemas \ac{ASR}, o \ac{RVA} funcionan
    bajo dominios muy acotados y no permiten al interlocutor desviarse de la
    entrada que espera el sistema, \eg \emph{``Por favor, diga ahora la opción a
      elegir: 1 Para\dots, 2 para\dots''}
  \item[Clasificación de documentos:] Una de las áreas más exitosas del
    \ac{NLP}, cuyo objetivo es identificar a qué categoría debería pertenecer un
    documento. Ha demostrado tener un amplio abanico de aplicaciones, \eg
    filtrado de \emph{spam}, clasificación de artículos de noticias,
    valoraciones de películas\dots Parte de su éxito e impacto se debe a la
    facilidad relativa que conlleva entrenar los modelos de aprendizaje para
    hacer dichas clasificaciones.
\end{description}

El \ac{NLP} emplea técnicas computacionales con el propósito de aprender,
entender y producir lenguaje humano. Las aproximaciones de hace unos años en el
campo de la investigación del lenguaje se centraban en automatizar el análisis
de las estructuras lingüísticas y desarrollar tecnologías como las mencionadas
anteriormente. Los investigadores de hoy en día se centran en usar dichas
herramientas en aplicaciones para el mundo real, creando sistemas de diálogo
hablados y motores de traducción \emph{Speech-to-Speech}, es decir, dados dos
interlocutores, interpretar y traducir sus frases. Otro de los focos en los que
se centran las investigaciones actuales son la minería en redes sociales en
busca de información sobre salud, finanzas e identificar los sentimietos y
emociones sobre determinados productos. 

\section{Historia del Procesamiento del Lenguaje Natural}
\label{sec:currentnlp}

A continuación, citamos algunos de los avances en este campo durante los últimos
años según \citeauthor{Hirschberg261} \citep{Hirschberg261}.

Durante las primeras épocas de esta ciencia, se intentaron escribir vocabularios
y reglas del lenguaje humano para que el ordenador las entendiera. Sin embargo,
debido a la naturaleza ambigua, variable e interpretación dependiente del
contexto de nuestro lenguaje resultó una ardua tarea. Por ejemplo, una estrella
puede ser un ente astronómico o una persona, y puede ser un nombre o un verbo.

En la década de los 90, los investigadores transformaron el mundo del \ac{NLP}
desarrollando modelos sobre grandes cantidades de datos sobre lenguajes. Estas
bases de datos se conocen como \emph{corpus}. 