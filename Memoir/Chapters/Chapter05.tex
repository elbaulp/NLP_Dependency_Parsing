%************************************************
\chapter{Implementación}
\label{ch:impl}
%************************************************

En este capítulo se detallan los aspectos de la implementación en \textsc{Scala}
del algoritmo introducido en el \autoref{ch:algorithm}. Así mismo, las ventajas
del desarrollo en \textsc{scala} pueden consultarse en el \autoref{ch:scalaintro}

\section{Planificación}
\label{sec:planning}

\question[inline]{Hay que poner algo?}

\section{Análisis y Diseño}
\label{sec:design}


\subsection{Diagrama de Clases}
\label{subsec:classdiagram}

Para el desarrollo del parseador de dependencias, se ha ideado el diseño de la
\autoref{fig:classdiag}
\begin{figure}[ht]
  \centering
  \tikzumlset{fill package=gray!20, fill class=gray!20}
  \begin{tikzpicture}
    \umlsimpleclass[x=0,y=1.5]{Trait TrainSentence}
    \umlsimpleclass[x=4,y=1.5]{Trait TestSentence}

    \umlsimpleclass[x=5.5,y=0]{Sentence}
    \umlsimpleclass[x=0,y=0]{LabeledSentence}

    \umlVHVinherit{Sentence}{Trait TestSentence}
    \umlVHVinherit{LabeledSentence}{Trait TestSentence}
    
    % \umlinherit[x=1,y=0,geometry=-|]{Sentence}{Trait TestSentence}
    % \umlinherit[x=4,y=0,geometry=-|]{LabeledSentence}{Trait TestSentence}
    \umlinherit[x=4,y=0]{LabeledSentence}{Trait TrainSentence}
  \end{tikzpicture}
  \missingfigure{TODO}
  \caption{Diagrama de clases \label{fig:classdiag}}
\end{figure}




\section{Implementación y Pruebas}
\label{sec:impl}



%*****************************************
%*****************************************
%*****************************************
%*****************************************
%*****************************************
