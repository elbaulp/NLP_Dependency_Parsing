% $Header: /Users/joseph/Documents/LaTeX/beamer/solutions/conference-talks/conference-ornate-20min.en.tex,v 90e850259b8b 2007/01/28 20:48:30 tantau $

\documentclass{beamer}

% This file is a solution template for:

% - Talk at a conference/colloquium.
% - Talk length is about 20min.
% - Style is ornate.

% Copyright 2004 by Till Tantau <tantau@users.sourceforge.net>.
%
% In principle, this file can be redistributed and/or modified under
% the terms of the GNU Public License, version 2.
%
% However, this file is supposed to be a template to be modified
% for your own needs. For this reason, if you use this file as a
% template and not specifically distribute it as part of a another
% package/program, I grant the extra permission to freely copy and
% modify this file as you see fit and even to delete this copyright
% notice. 


\mode<presentation>
{
  \usetheme{Berkeley}
  \usefonttheme[onlylarge]{structuresmallcapsserif}
  
  \setbeamertemplate{navigation symbols}{}
  \setbeamertemplate{caption}[numbered]
  \setbeamertemplate{footline}[page number]

  % or ...

  \setbeamercovered{transparent}
  % or whatever (possibly just delete it)
}

\usepackage[spanish]{babel}
% or whatever

\usepackage[latin1]{inputenc}
% or whatever

\usepackage{times}
\usepackage[T1]{fontenc}
% Or whatever. Note that the encoding and the font should match. If T1
% does not look nice, try deleting the line with the fontenc.

\usepackage{tikz}      % Diagrams
\usetikzlibrary{calc, shapes, backgrounds,shapes.geometric,arrows,decorations.markings,babel,positioning,calc}
\usepackage{booktabs}  % Tables
\usepackage{tabularx}
\usepackage{algorithm}% http://ctan.org/pkg/algorithms
\usepackage{algpseudocode}% http://ctan.org/pkg/algorithmicx
\usepackage{subcaption}

\usepackage[outputdir=metafiles]{minted}
\usemintedstyle{tango}
\usepackage{pgfgantt}

\newminted{scala}{
   fontsize=\footnotesize,
   autogobble=true,
   frame=lines,
   mathescape=true,
   breaklines,
   breakautoindent,
   mathescape
}

\newminted{java}{
   fontsize=\footnotesize,
   autogobble=true,
   frame=lines,
   mathescape=true,
   breaklines,
   breakautoindent,
   mathescape
}

\title[Dep Parsing Castellano]%
{DISEÑO E IMPLEMENTACIÓN DE UN ANALIZADOR DE DEPENDENCIAS PARA PROCESAMIENTO DE LENGUAJE NATURAL EN ESPAÑOL} %% title page.

% \subtitle
% {Include Only If Paper Has a Subtitle}

\author[A. Alcalde] % (optional, use only with lots of authors)
{Alejandro Alcalde~\inst{1}}
% - Give the names in the same order as the appear in the paper.
% - Use the \inst{?} command only if the authors have different
%   affiliation.

\institute[ETSIIT] % (optional, but mostly needed)
{
  \inst{1}%
  Grado Ingeniería Informática\\
  Universidad de Granada
}
% - Use the \inst command only if there are several affiliations.
% - Keep it simple, no one is interested in your street address.

\date[Diciembre 2016]{\today}
% - Either use conference name or its abbreviation.
% - Not really informative to the audience, more for people (including
%   yourself) who are reading the slides online

\subject{Procesamiento Lenguaje Natural}
% This is only inserted into the PDF information catalog. Can be left
% out. 


% If you have a file called "university-logo-filename.xxx", where xxx
% is a graphic format that can be processed by latex or pdflatex,
% resp., then you can add a logo as follows:

% \pgfdeclareimage[height=0.5cm]{university-logo}{udg}
% \logo{\pgfuseimage{university-logo}}

% Delete this, if you do not want the table of contents to pop up at
% the beginning of each subsection:
\AtBeginSubsection[]
{
  \begin{frame}<beamer>{Índice}
    \tableofcontents[currentsection,currentsubsection]
  \end{frame}
}

% If you wish to uncover everything in a step-wise fashion, uncomment
% the following command: 

%\beamerdefaultoverlayspecification{<+->}


\begin{document}

\begin{frame}
  \titlepage
\end{frame}

\begin{frame}{Índice}
  \tableofcontents
  % You might wish to add the option [pausesections]
\end{frame}


% Structuring a talk is a difficult task and the following structure
% may not be suitable. Here are some rules that apply for this
% solution: 

% - Exactly two or three sections (other than the summary).
% - At *most* three subsections per section.
% - Talk about 30s to 2min per frame. So there should be between about
%   15 and 30 frames, all told.

% - A conference audience is likely to know very little of what you
%   are going to talk about. So *simplify*!
% - In a 20min talk, getting the main ideas across is hard
%   enough. Leave out details, even if it means being less precise than
%   you think necessary.
% - If you omit details that are vital to the proof/implementation,
%   just say so once. Everybody will be happy with that.

\section{Motivación}

\subsection{Falta de Software Español}

\definecolor{tick}{HTML}{5dc452}
\newcommand*{\checktikz}[1][]{\tikz[x=1em, y=1em]\fill[#1] (0,.35) -- (.25,0) --
  (1,.7) -- (.25,.15) -- cycle;}
\newcommand*{\ccheck}{\checktikz[tick,rounded corners=.5pt, draw=tick,
  thin]} %\checktikz[rounded corners=.5pt, draw=red, ultra thin]

\begin{frame}{Soporte de Idiomas en Pipelines Actuales}
  % - A title should summarize the slide in an understandable fashion
  %   for anyone how does not follow everything on the slide itself.
  \begin{table}[!b]
    \begin{tabularx}{\textwidth}{lllllll}
      \textbf{ANNOTATOR} & \textbf{AR} & \textbf{ZH} & \textbf{EN} & \textbf{FR} & \textbf{DE} & \textbf{ES} \\
      \toprule
      Tokenize/Segment & \ccheck & \ccheck & \ccheck & \ccheck &  & \ccheck \\
      Sentence Split & \ccheck & \ccheck & \ccheck & \ccheck & \ccheck & \ccheck \\
      Part of Speech & \ccheck & \ccheck & \ccheck & \ccheck & \ccheck & \ccheck \\
      Lemma &  &  & \ccheck &  &  &  \\
      Named Entities &  & \ccheck & \ccheck &  & \ccheck & \ccheck \\
      Constituency Parsing & \ccheck & \ccheck & \ccheck & \ccheck & \ccheck & \ccheck \\
      Dependency Parsing &  & \ccheck & \ccheck & \ccheck & \ccheck &  \\
      Sentiment Analysis &  &  & \ccheck &  &  &  \\
      Mention Detection &  & \ccheck & \ccheck &  &  &  \\
      Coreference &  & \ccheck & \ccheck &  &  &  \\
      Open IE &  &  & \ccheck &  &  & \\
      \bottomrule
    \end{tabularx}
  \end{table}
\end{frame}

\subsection{Introducción al NLP}

\begin{frame}{Qué es el NLP}
  \begin{block}{Definición}
    Ciencia que estudia la computación lingüística.
  \end{block}
  \begin{itemize}
  \item Resúmenes.
  \item Traducción automática.
  \item Reconocimiento de voz.
  \item Sistemas de Diálogo Hablado.
  \item Clasificación de documentos.
  \item Análisis de sentimientos.
  \end{itemize}
\end{frame}

\begin{frame}{Niveles de Análisis}
  \begin{itemize}
  \item Documento.
  \item Sentencia.
  \item Entidad.
  \end{itemize}
\end{frame}

\section{Objetivos}
\begin{frame}{Objetivos}
  \begin{itemize}
  \item Revisión bibliográfica.
  \item Elección de un parseador y diseño para \textsc{Scala}.
  \item Implementación y TDD.
  \item Evaluación y comparación de resultados.
  \end{itemize}
\end{frame}

\section{Resolución del Trabajo}

\subsection{Algoritmo}

\tikzset
{common/.style =
  {rectangle, rounded corners, minimum width=1cm, minimum height=1cm,
    text centered, text opacity=1, align=center
  }
  ,notarget/.style = {common, draw=none, opacity=0, text opacity=0.5}
  ,target/.style   = {common, very thick, draw}
  ,blank/.style    = {common, draw=none}
}

\begin{frame}[label=alg]{Algoritmo Seleccionado para Español}
  \framesubtitle{Statistical Dependency Analysis With SVMs}
  \footnotesize
  \algdef{SnE}{Init}{EndInit}{\textbf{Initialize:}}
  \begin{algorithmic}[1] % The number tells where the line numbering should start
    \State \textbf{Input Sentence:} $(w_1, p_1),(w_2,p_2),\cdots,(w_n,p_n)$
    \Init
       \State $i\gets 1$
       \State ${\cal T}\gets \{(w_1, p_1),(w_2,p_2),\cdots,(w_n,p_n)\}$
       \State $\text{no\_construction}\gets \text{true}$
    \EndInit
       \While{$|{\cal T}| \geq 1$}
          \If{$i == |{\cal T}|$}
             \If{no\_construction == true}
                \textbf{break}
             \EndIf
             \State $\text{no\_construction}\gets \text{true}$
             \State $i\gets 1$
          \Else
             \State $\mathbf{x}\gets \text{getContextualFeatures(${\cal T}, i$)} $
             \State $y\gets \text{estimateAction(model, $\mathbf{x}$)}$
             \State construction(${\cal T}, i, y$)
             \If{$y == \text{Left or Right}$}
                $\text{no\_construction}\gets \text{false}$
             \EndIf
          \EndIf
       \EndWhile
  \end{algorithmic}
\end{frame}

\begin{frame}{Acción Desplazar}
  \begin{figure}[ht]
    \begin{columns}[onlytextwidth]
      \begin{column}{.5\textwidth}
        \begin{subfigure}{.5\textwidth}
          \begin{tikzpicture}[node distance=1.1cm]
            \node (n1) [notarget] {I\\\textsc{prp}};
            \node (n2) [target, right of=n1] {saw\\\textsc{vbd}};
            \node (n3) [target, right of=n2] {a\\\textsc{dt}};
            \node (n4) [notarget, right of=n3] {girl\\\textsc{nn}};
            \draw [thick,->] (n4) -- ++(1cm,0) node[sloped,above,midway] {\scriptsize\alert{\textsc{Shift}}};
          \end{tikzpicture}
          \caption{}
        \end{subfigure}
      \end{column}
      \begin{column}{.5\textwidth}
        \begin{subfigure}{.5\textwidth}
          \begin{tikzpicture}[node distance=1.1cm]
            \node (n1) [notarget] {I\\\textsc{prp}};
            \node (n2) [notarget, right of=n1] {saw\\\textsc{vbd}};
            \node (n3) [target, right of=n2] {a\\\textsc{dt}};
            \node (n4) [target, right of=n3] {girl\\\textsc{nn}};
          \end{tikzpicture}
          \caption{}
        \end{subfigure}
      \end{column}
    \end{columns}
    \caption{\textsc{Desplazar}. (a) Antes. (b) Después}
  \end{figure}
\end{frame}

\begin{frame}[label=right]{Acción Derecha}
  \begin{figure}[ht]
    \subcaptionbox{}{
      \begin{tikzpicture}[node distance=.5mm]
        \node (n1) [notarget] {I\\\textsc{prp}};
        \node (n2) [notarget, right=of n1] {saw\\\textsc{vbd}};
        \node (n3) [target, right=of n2] {a\\\textsc{dt}};
        \node (n4) [target, right=of n3] {girl\\\textsc{nn}};
        \draw [thick,->] ($(n4.east) + (5mm,0)$) -- ++(.5cm,0) node[above,midway] {\scriptsize\alert{\textsc{Right}}};
        \node[target,opacity=0,below=5mm of n4] {}; % <- Pseudonode
      \end{tikzpicture}
    }
    \subcaptionbox{}{
      \begin{tikzpicture}[node distance=.5mm]
        \node (n1) [notarget] {I\\\textsc{prp}};
        \node (n2) [notarget, right=of n1] {saw\\\textsc{vbd}};
        \node (non)[blank,right=of n2]{};
        \node (n3) [target, right=of non] {girl\\\textsc{nn}};
        \node (n4) [target, below=5mm of n3,anchor=north] {a\\\textsc{dt}};
        \draw [thick,->] (n4) -- (n3);
      \end{tikzpicture}
    }
    \caption{\textsc{Derecha}. (a) Antes. (b) Después.}
  \end{figure}
\end{frame}

\begin{frame}{Acción Izquierda}
  \begin{figure}[ht]
    \subcaptionbox{}{
      \begin{tikzpicture}[node distance=.5mm,baseline=(n3)]
        \node (n1) [notarget] {I\\\textsc{prp}};
        \node (n2) [target, right=of n1] {saw\\\textsc{vbd}};
        \node (n3) [target, right=of n2] {girl\\\textsc{nn}};
        \node (n4) [notarget, below=5mm of n3] {a\\\textsc{dt}};
        \draw [thick,->] (n4) -- (n3);
        \draw [thick,->] ($(n3.east) + (1cm,0)$) -- ++(.5cm,0) node[above,midway] {\scriptsize\alert{\textsc{Left}}};
        \node[target,opacity=0,below=5mm of n4] {}; % <- Pseudonode
      \end{tikzpicture}
    }
    \subcaptionbox{}{
      \begin{tikzpicture}[node distance=.5mm,baseline=(n1)]
        \node (n1) [notarget] {I\\\textsc{prp}};
        \node (n2) [target, right=of n1] {saw\\\textsc{vbd}};
        \node (n4) [target, below=5mm of n2] {girl\\\textsc{nn}};
        \node (n3) [notarget, below=5mm of n4] {a\\\textsc{dt}};
        \draw [thick,->] (n4) -- (n2);
        \draw [thick,->] (n3) -- (n4);
      \end{tikzpicture}
    }
    \caption{\textsc{Izquierda}. (a) Antes. (b) Después}
  \end{figure}
\end{frame}

\begin{frame}{Ejemplo -- ``\emph{Sobre la oferta de IBM}''}
  \begin{figure}[ht]
  \footnotesize
  \setlength{\extrarowheight}{16pt}
  \begin{tabular}{p{1\textwidth}}
    \begin{tikzpicture}[node distance=1mm,baseline=(sobre)]
      \node (sobre) [target] {Sobre\\\textsc{adp}};
      \node (la) [target, right=of sobre] {la\\\textsc{det}};
      \node (oferta) [notarget, right=of la] {oferta\\\textsc{noun}};
      \node (de) [notarget, right=of oferta] {de\\\textsc{adp}};
      \node (inter) [notarget, right=of de] {IBM\\\textsc{noun}};
      \node (b1) [blank,right=of inter]{};
      \draw [thick,->] (b1.west) -- ++(.5cm,0) node[above,midway]{\tiny\alert{\textsc{Shift}}};
    \end{tikzpicture}
\\
   \begin{tikzpicture}[node distance=1mm,baseline=(sobre)]
     \node (sobre) [notarget] {Sobre\\\textsc{adp}};
     \node (la) [target, right=of sobre] {la\\\textsc{det}};
     \node (oferta) [target, right=of la] {oferta\\\textsc{noun}};
     \node (de) [notarget, right=of oferta] {de\\\textsc{adp}};
     \node (inter) [notarget, right=of de] {IBM\\\textsc{noun}};
     \node (b1) [blank,right=of inter]{};
   \end{tikzpicture}

  \end{tabular}
\end{figure}
\end{frame}

\begin{frame}{Ejemplo -- ``\emph{Sobre la oferta de IBM}''}  
  \begin{figure}[ht]
    \footnotesize
    \setlength{\extrarowheight}{16pt}
    \begin{tabular}{p{1\textwidth}}
    \begin{tikzpicture}[node distance=1mm,baseline=(sobre)]
      \node (sobre) [notarget] {Sobre\\\textsc{adp}};
      \node (la) [target, right=of sobre] {la\\\textsc{det}};
      \node (oferta) [target, right=of la] {oferta\\\textsc{noun}};
      \node (de) [notarget, right=of oferta] {de\\\textsc{adp}};
      \node (inter) [notarget, right=of de] {IBM\\\textsc{noun}};
      \node (b1) [blank,right=of inter]{};
      \draw [thick,->] (b1.west) -- ++(.5cm,0) node[above,midway]{\tiny\alert{\textsc{Right}}};
    \end{tikzpicture}
\\
   \begin{tikzpicture}[node distance=1mm,baseline=(sobre)]
     \node (sobre) [target] {Sobre\\\textsc{adp}};
     \node (b1) [blank,right=of sobre] {};
     \node (la) [notarget,below=.3cm of oferta] {la\\\textsc{det}};
     \node (oferta) [target, right=of b1] {oferta\\\textsc{noun}};
     \node (de) [notarget, right=of oferta] {de\\\textsc{adp}};
     \node (inter) [notarget, right=of de] {IBM\\\textsc{noun}};
     \node (b1) [blank,right=of inter]{};
     \draw [thick,->] (la) -- (oferta);
   \end{tikzpicture}
    \end{tabular}
  \end{figure}
\end{frame}

\begin{frame}{Ejemplo -- ``\emph{Sobre la oferta de IBM}''}  
  \begin{figure}[ht]
    \footnotesize
    \setlength{\extrarowheight}{5pt}
    \begin{tabular}{p{1\textwidth}}
    \begin{tikzpicture}[node distance=1mm,baseline=(sobre)]
      \node (sobre) [target] {Sobre\\\textsc{adp}};
      \node (b1) [blank,right=of sobre] {};
      \node (la) [notarget,below=.3cm of oferta] {la\\\textsc{det}};
      \node (oferta) [target, right=of b1] {oferta\\\textsc{noun}};
      \node (de) [notarget, right=of oferta] {de\\\textsc{adp}};
      \node (inter) [notarget, right=of de] {IBM\\\textsc{noun}};
      \node (b1) [blank,right=of inter]{};

      \draw [thick,->] (la) -- (oferta);
      \draw [thick,->] (b1.west) -- ++(.5cm,0) node[above,midway]{\tiny\alert{\textsc{Right}}};
    \end{tikzpicture}
\\
   \begin{tikzpicture}[node distance=1mm,baseline=(b2)]
     \node (b2) [blank] {};
     \node (b1) [blank,right=of b2] {};
     \node (la) [notarget,below=.3cm of oferta] {la\\\textsc{det}};
     \node (oferta) [target, right=of b1] {oferta\\\textsc{noun}};
     \node (sobre) [notarget,below=.3cm of la] {Sobre\\\textsc{adp}};
     \node (de) [target, right=of oferta] {de\\\textsc{adp}};
     \node (inter) [notarget, right=of de] {IBM\\\textsc{noun}};
     \node (b1) [blank,right=of inter]{};

     \draw [thick,->] (la) -- (oferta);
     \draw [thick,->] (sobre) -- (la);
   \end{tikzpicture}
    \end{tabular}
  \end{figure}
\end{frame}

\begin{frame}{Ejemplo -- ``\emph{Sobre la oferta de IBM}''}  
  \begin{figure}[ht]
    \footnotesize
    \setlength{\extrarowheight}{5pt}
    \begin{tabular}{p{1\textwidth}}
    \begin{tikzpicture}[node distance=1mm,baseline=(b2)]
      \node (b2) [blank] {};
      \node (b1) [blank,right=of b2] {};
      \node (la) [notarget,below=.3cm of oferta] {la\\\textsc{det}};
      \node (oferta) [target, right=of b1] {oferta\\\textsc{noun}};
      \node (sobre) [notarget,below=.3cm of la] {Sobre\\\textsc{adp}};
      \node (de) [target, right=of oferta] {de\\\textsc{adp}};
      \node (inter) [notarget, right=of de] {IBM\\\textsc{noun}};
      \node (b1) [blank,right=of inter]{};

      \draw [thick,->] (la) -- (oferta);
      \draw [thick,->] (sobre) -- (la);

      \draw [thick,->] (b1.west) -- ++(.5cm,0) node[above,midway]{\tiny\alert{\textsc{Shift}}};
    \end{tikzpicture}
\\
    \begin{tikzpicture}[node distance=1mm,baseline=(b2)]
      \node (b2) [blank] {};
      \node (b1) [blank,right=of b2] {};
      \node (la) [notarget,below=.3cm of oferta] {la\\\textsc{det}};
      \node (oferta) [notarget, right=of b1] {oferta\\\textsc{noun}};
      \node (sobre) [notarget,below=.3cm of la] {Sobre\\\textsc{adp}};
      \node (de) [target, right=of oferta] {de\\\textsc{adp}};
      \node (inter) [target, right=of de] {IBM\\\textsc{noun}};
      \node (b1) [blank,right=of inter]{};

      \draw [thick,->] (la) -- (oferta);
      \draw [thick,->] (sobre) -- (la);
    \end{tikzpicture}
    \end{tabular}
  \end{figure}
\end{frame}

\begin{frame}{Ejemplo -- ``\emph{Sobre la oferta de IBM}''}  
  \begin{figure}[ht]
    \footnotesize
    \setlength{\extrarowheight}{1pt}
    \begin{tabular}{p{1\textwidth}}
    \begin{tikzpicture}[node distance=1mm,baseline=(b2)]
      \node (b2) [blank] {};
      \node (b1) [blank,right=of b2] {};
      \node (la) [notarget,below=.3cm of oferta] {la\\\textsc{det}};
      \node (oferta) [notarget, right=of b1] {oferta\\\textsc{noun}};
      \node (sobre) [notarget,below=.3cm of la] {Sobre\\\textsc{adp}};
      \node (de) [target, right=of oferta] {de\\\textsc{adp}};
      \node (inter) [target, right=of de] {IBM\\\textsc{noun}};
      \node (b1) [blank,right=of inter]{};

      \draw [thick,->] (la) -- (oferta);
      \draw [thick,->] (sobre) -- (la);

      \draw [thick,->] (b1.west) -- ++(.5cm,0) node[above,midway]{\tiny\alert{\textsc{Right}}};
    \end{tikzpicture}
\\
       \begin{tikzpicture}[node distance=1mm,baseline=(b2)]
      \node (b2) [blank] {};
      \node (b1) [blank,right=of b2] {};
      \node (la) [notarget,below=.3cm of oferta] {la\\\textsc{det}};
      \node (oferta) [target, right=of b1] {oferta\\\textsc{noun}};
      \node (sobre) [notarget,below=.3cm of la] {Sobre\\\textsc{adp}};
      \node (de) [notarget, below=.3cm of inter] {de\\\textsc{adp}};
      \node (b3) [blank,right=of oferta] {};
      \node (inter) [target, right=of b3] {IBM\\\textsc{noun}};

      \draw [thick,->] (la) -- (oferta);
      \draw [thick,->] (sobre) -- (la);
      \draw [thick,->] (de) -- (inter);
    \end{tikzpicture}
    \end{tabular}
  \end{figure}
\end{frame}

\begin{frame}{Selección de Características}
  \begin{table}[b!]
    \begin{tabularx}{\textwidth}{lp{.8\textwidth}}
      \toprule
      Tipo     & Valor                                                                \\
      \toprule
      pos      & POS \emph{tag}                                                       \\
      lex      & La palabra                                                           \\
      ch-L-pos & Nodo hijo modificando al padre por la izda.                          \\
      ch-L-lex & Palabra del correspondiente ch-L-pos                                 \\
      ch-R-pos & Nodo hijo modificando al padre por la drcha                          \\
      ch-R-lex & Palabra del correspondiente ch-R-pos                                 \\
      \bottomrule
    \end{tabularx}
  \end{table}  
\end{frame}

\subsection{Resultados}

\begin{frame}{Resultados}
  \begin{table}[b!]
    \begin{tabularx}{.82\textwidth}{p{.5\textwidth}|p{.1\textwidth}p{.1\textwidth}}
      Kernel: $(x'\cdot x'' + 1)^2$, Ctx: $(2,4)$ & TFG & ROHIT \\
      \toprule
      Dep. Acc.  & 76\%   & 75\% \\
      Root Acc.  & 67\%   & 70\% \\
      Comp. Rate & 15\%   & 11\% \\
      \bottomrule
    \end{tabularx}
  \end{table}
\end{frame}


\subsection{Implementación}

\begin{frame}{Implementación}{Planificación}
  \definecolor{barblue}{RGB}{153,204,254}
  \definecolor{groupblue}{RGB}{51,102,254}
  \definecolor{linkred}{RGB}{165,0,33}
  \renewcommand\sfdefault{phv}
  \renewcommand\mddefault{mc}
  \renewcommand\bfdefault{bc}
  \setganttlinklabel{s-s}{START-TO-START}
  \setganttlinklabel{f-s}{FINISH-TO-START}
  \setganttlinklabel{f-f}{FINISH-TO-FINISH}
  \noindent\resizebox{\textwidth}{!}{
  \begin{tikzpicture}[x=0cm, y=0cm]
\begin{ganttchart}[
  canvas/.append style={fill=none, draw=black!5, line width=.75pt},
  hgrid style/.style={draw=black!5, line width=.75pt},
  vgrid={*1{draw=black!5, line width=.75pt}},
  today=14,
  today rule/.style={
    draw=black!64,
    dash pattern=on 3.5pt off 4.5pt,
    line width=1.5pt
  },
  today label font=\small\scshape,
  title/.style={draw=none, fill=none},
  title label font=\scshape\footnotesize,
  title label node/.append style={below=7pt},
  include title in canvas=false,
  bar label font=\mdseries\small\color{black!70},
  bar label node/.append style={left=2cm},
  bar/.append style={draw=none, fill=black!63},
  bar incomplete/.append style={fill=barblue},
  bar progress label font=\mdseries\footnotesize\color{black!70},
  group incomplete/.append style={fill=groupblue},
    group left shift=0,
    group right shift=0,
    group height=.5,
    group peaks tip position=0,
    group label node/.append style={left=.6cm},
    group progress label font=\scshape\small,
    link/.style={-latex, line width=1.5pt, linkred},
    link label font=\scriptsize\scshape,
    link label node/.append style={below left=-2pt and 0pt},
  ]{1}{14}
  [grid]
  \gantttitle{\tiny Septiembre}{4}
  \gantttitle{\tiny Octubre}{4} 
  \gantttitle{\tiny Noviembre}{4}
  \gantttitle{\tiny Diciembre}{2}\\
  \gantttitle[
    title label node/.append style={below left=7pt and -3pt}
  ]{Semana:\quad1}{1}
  \gantttitlelist{2,...,14}{1} \\
  \ganttgroup[progress=100]{Progreso}{1}{14} \\
  \ganttbar[
    progress=100,
    name=research
  ]{Investigación}{1}{4} \\
  \ganttbar[
    progress=100,
    name=design
  ]{Análisis y Diseño}{5}{5} \\
  \ganttbar[
    progress=100,
    name=impl
  ]{Implementación}{6}{11} \\
  \ganttbar[
    progress=100,
    name=memoir
  ]{Memoria}{12}{14} \\    
  
  \ganttmilestone{M1: Conocer el campo del NLP}{4}{4}  \\
  \ganttmilestone{M2: Finalizar Código}{11}{11} \\
  \ganttmilestone{M3: Finalización TFG}{14}{14}
  
  \ganttlink[link type=f-s]{research}{design}
  \ganttlink[link type=f-s]{design}{impl}
  \ganttlink[link type=f-s]{impl}{memoir}
\end{ganttchart}
\end{tikzpicture}}
\end{frame}

\begin{frame}{Implementación}{Por Qué en \textsc{Scala}}
  \begin{columns}[onlytextwidth,c]
    \column{.5\textwidth}
    \begin{itemize}
    \item Programación OO.
    \item Programación Funcional.
    \item Sintaxis breve.
    \item Escalable.
    \end{itemize}
    \column{.5\textwidth}
    \begin{itemize}
    \item Implementa algunos patrones.
    \item \textsc{Traits}.
    \item Amplio abanico reglas de visibilidad.
    \end{itemize}
  \end{columns}
\end{frame}

\begin{frame}[fragile]{Implementación}{Ventajas de \textsc{Scala}}
  \begin{columns}[onlytextwidth]
    \begin{column}{.5\textwidth}
      \begin{javacode*}{fontsize=\fontsize{5.2}{4},label=\textsc{Java}}
    class WordCountMapper extends MapReduceBase
    implements Mapper<IntWritable, Text, Text, IntWritable> {

      static final IntWritable one  = new IntWritable(1);
      // Value will be set in a non-thread-safe way!
      static final Text word = new Text;

      @Override
      public void map(IntWritable key, Text valueDocContents,
      OutputCollector<Text, IntWritable> output, Reporter reporter) {
        String[] tokens = valueDocContents.toString.split("\\s+");       
        for (String wordString: tokens) {
          if (wordString.length > 0) {
            word.set(wordString.toLowerCase);
            output.collect(word, one);
          }
        }
      }
    }

    class WordCountReduce extends MapReduceBase
    implements Reducer<Text, IntWritable, Text, IntWritable> {

      public void reduce(Text keyWord, java.util.Iterator<IntWritable> counts,
      OutputCollector<Text, IntWritable> output, Reporter reporter) {
        int totalCount = 0;
        while (counts.hasNext) {                                         
          while (counts.hasNext) {
            totalCount += counts.next.get;
          }
          output.collect(keyWord, new IntWritable(totalCount));
        }
      }
      \end{javacode*}
    \end{column}
    \begin{column}{.5\textwidth}
      \begin{scalacode*}{fontsize=\fontsize{5.2}{4},label=\textsc{Scala}}
    class ScaldingWordCount(args : Args) extends Job(args) {
      TextLine(args("input"))
      .read
      .flatMap('line -> 'word) {
        line: String => line.trim.toLowerCase.split("""\s+""")
      }
      .groupBy('word){ group => group.size('count) }
      .write(Tsv(args("output")))                   
    }
      \end{scalacode*}
    \end{column}
  \end{columns}
\end{frame}

\begin{frame}{Metodología: TDD y BDD}
  \begin{columns}[onlytextwidth,t]
    \column{.5\textwidth}
    \alert{Test-Driven\\ Development}
    \begin{itemize}
    \item Rojo.
    \item Verde.
    \item Refactorizar.
    \end{itemize}
    \column{.5\textwidth}
    \alert{Behavior-Driven\\ Development}
    \begin{itemize}
    \item \emph{Given}.
    \item \emph{When}.
    \item \emph{Then}.
    \end{itemize}
  \end{columns}
\end{frame}

\begin{frame}{TDD para AA}
  \begin{block}{Filosofía}
    Aplicar \alert{TDD} a problemas de Aprendizaje
  \end{block}
  \begin{columns}[onlytextwidth,t]
    \column{.5\textwidth}
    \begin{itemize}
    \item \textsc{R2} \emph{Value}.
    \item ROC y AUC.
    \item Matriz confusión.
    \end{itemize}
    \column{.5\textwidth}
    \begin{itemize}
    \item Establecer un \alert{baseline}.
    \item Intentar mejorarlo \\en cada iteración.
    \end{itemize}
  \end{columns}
\end{frame}

\begin{frame}[fragile]{TDD para AA}
  \framesubtitle{Ejemplo}
    \begin{scalacode*}{fontsize=\tiny}
class DependencyParserCheckBaselineSpec extends Specification
  with GWT
  with StandardRegexStepParsers {def is = s2"""
  When training the model, set the following baselines  ${featuresBaseline.start}
    Given Train data set: es_ancora-converted-train1
    Given Test data set: es_ancora-converted-test1
    When Genenaring Vocabulary
    Then Dep. Acc should be at least: 70%
    and Root Acc should be at least: 50%
    and Comp. Acc should be at least: 3%                ${featuresBaseline.end}
"""
// ...    
    \end{scalacode*}

\end{frame}

\subsection[Conclusiones]{Conclusiones y Vías Futuras}
\begin{frame}{Conclusiones y Vías Futuras}
  \begin{columns}[onlytextwidth,t]
    \column{.5\textwidth}
    Conclusiones
    \begin{itemize}
    \item Implementado parseo\\ para \alert{Castellano}.
    \item Uso de \alert{SVMs}.
    \item \textsc{Scala}.
    \item \alert{TDD} para AA.
    \end{itemize}
    \column{.5\textwidth}
    Vías futuras
    \begin{itemize}
    \item Más algoritmos.
    \item Código 100\% funcional.
    \item Más fases del \alert{Pipeline}.
    \item \textsc{Spark}.
    \end{itemize}
  \end{columns}
\end{frame}

\begin{frame}{Fin}
  Código disponible:\\\url{https://github.com/algui91/NLP_Dependency_Parsing}
  \bigskip

  Muchas gracias por su atención.

  --- Alejandro Alcalde.
\end{frame}

\section*{Summary}

\begin{frame}{Summary}

  % Keep the summary *very short*.
  \begin{itemize}
  \item
    The \alert{first main message} of your talk in one or two lines.
  \item
    The \alert{second main message} of your talk in one or two lines.
  \item
    Perhaps a \alert{third message}, but not more than that.
  \end{itemize}
  
  % The following outlook is optional.
  \vskip0pt plus.5fill
  \begin{itemize}
  \item
    Outlook
    \begin{itemize}
    \item
      Something you haven't solved.
    \item
      Something else you haven't solved.
    \end{itemize}
  \end{itemize}
\end{frame}



% All of the following is optional and typically not needed. 
% \appendix
% \section<presentation>*{\appendixname}
% \subsection<presentation>*{For Further Reading}

% \begin{frame}[allowframebreaks]
%   \frametitle<presentation>{For Further Reading}
    
%   \begin{thebibliography}{10}
    
%   \beamertemplatebookbibitems
%   % Start with overview books.

%   \bibitem{Author1990}
%     A.~Author.
%     \newblock {\em Handbook of Everything}.
%     \newblock Some Press, 1990.
 
    
%   \beamertemplatearticlebibitems
%   % Followed by interesting articles. Keep the list short. 

%   \bibitem{Someone2000}
%     S.~Someone.
%     \newblock On this and that.
%     \newblock {\em Journal of This and That}, 2(1):50--100,
%     2000.
%   \end{thebibliography}
% \end{frame}

\end{document}
